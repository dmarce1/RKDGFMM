\documentclass[preprint]{aastex}
\usepackage{natbib}
\usepackage{morefloats}
\usepackage{amsmath}
\usepackage{graphicx}
\usepackage{verbatim}
\usepackage{color}
\usepackage{ulem}
\usepackage{appendix}
\usepackage{algpseudocode}
\usepackage{algorithmicx}
\bibliographystyle{apj}
\begin{document}

\newcommand{\fig}[1]{Figure \ref{#1}}
\newcommand{\eq}[1]{Equation (\ref{#1})}
\newcommand{\eqs}[2]{Equations (\ref{#1}) - (\ref{#2})}
\newcommand{\vc}[1]{ {\bf{#1}} }
\newcommand{\grad}[1]{ \vc{\nabla } #1 }
\newcommand{\dv}[1]{ \vc{\nabla } \cdot #1 }
\newcommand{\dt}[1]{ \frac{\partial}{\partial t} #1 }
\newcommand{\dx}[2]{ \frac{\partial}{\partial #1} #2 }
\newcount\colveccount
\newcommand*\colvec[1]{
        \global\colveccount#1
        \begin{pmatrix}
        \colvecnext
}
\def\colvecnext#1{
        #1
        \global\advance\colveccount-1
        \ifnum\colveccount>0
                \\
                \expandafter\colvecnext
        \else
                \end{pmatrix}
        \fi
}

\title{Octupus}
\author{Dominic C. Marcello}

\section{}

The general form for hyperbolic conservation laws is
\begin{equation}
\dt \vc{w} + \dv{\vc{f\left(\vc{w}\right)}} = \vc{s},
\end{equation}
where the matrix $\frac{\partial \vc{f\left(\vc{w}\right)}}{\partial \vc{w}}$ has real eigenvalues. For Euler's equations of incompressible ideal gas flow, 
\begin{equation}
\vc{w} = \colvec{3}{\rho}{\vc{s}}{\mathrm{E}},
\end{equation}
and
\begin{equation}
\vc{f\left(\vc{w}\right)} = \colvec{3}{\rho \vc{v} }{\vc{s} \vc{v} + \mathrm{p}}{\left(\mathrm{E} + \mathrm{p}\right) \vc{v}},
\end{equation}
where $\rho$ is the mass density, $\vc{s}$ is the momentum density, $\mathrm{E}$ is the energy density, 
\begin{equation}
\vc{v} = \frac{\vc{s}}{\rho}
\end{equation}
is the velocity, and 
\begin{equation}
p := \left(\gamma-1\right) (\mathrm{E} - \frac{1}{2} \rho u^2)
\end{equation}
is the gas pressure for a given ratio of specific heats, $\gamma$. For flow in conservative scalar potentials, such as classical gravity,
\begin{equation}
\vc{s} := \colvec{3}{0}{-\rho \grad{\phi}}{-\rho \vc{u} \cdot \grad{\phi}},
\end{equation}
where $\phi$ is the potential.

We use the positivty limiter of \cite{ZS2010}.

\bibliography{paper1}

\end{document}



